\documentclass{beamer}

% Theme choice
\usetheme{Madrid}

% Packages
\usepackage{graphicx}
\usepackage{hyperref}

% Title page details
\title{BiCycure: Tackling Bicycle Theft}
\subtitle{A Comprehensive Solution to Modernize Bicycle Theft resolution}
\author{Software Architecture - Team  28}
\institute{TU Delft}
\date{\today}

\begin{document}

% Title slide
\begin{frame}
    \titlepage
\end{frame}

% Table of Contents slide
\begin{frame}{Table of Contents}
    \tableofcontents
\end{frame}

% Problem Description
\section{Problem Description}
\begin{frame}{Problem Description}
    \begin{itemize}
        \item In 2023, 1.3 bicycles per person in the Netherlands.
        \item Between 468,000 and 772,000 bicycles stolen annually.
        \item Intentionally handling stolen goods is illegal with jail time or fines up to €87,000.
        \item The current process for checking if a bicycle is stolen is manual and inefficient.
    \end{itemize}
\end{frame}

% Existing Solutions
\section{Existing Solutions}
\begin{frame}{Stop Heling and RDW fietsendiefstalregister}
    \begin{block}{Stop Heling}
        \begin{itemize}
            \item Mobile app for users to register frame numbers and check against a database of stolen bikes.
            \item Does not allow theft reporting—users must report to police separately.
        \end{itemize}
    \end{block}
    
    \begin{block}{RDW fietsendiefstalregister}
        \begin{itemize}
            \item Public database for stolen bicycles.
            \item No mobile app.
        \end{itemize}
    \end{block}
\end{frame}

\begin{frame}{Problems with Existing Solutions}
    \begin{itemize}
        \item Manual entry of frame numbers by owners.
        \item No direct theft report feature in apps.
        \item Only manual frame number checks; little automation in theft recovery.
    \end{itemize}
\end{frame}

% Proposed Solution
\section{Proposed Solution}
\begin{frame}{BiCycure: A Modern Solution}
    \begin{itemize}
        \item RFID tags embedded in bicycles for tamper-proof identification.
        \item Mobile app for ownership verification, theft declaration, and transfer of ownership.
        \item Seamless ownership transfer between individuals with bilateral confirmation.
        \item Community-driven recovery with public theft reports and rewards.
    \end{itemize}
\end{frame}

% Features of BiCycure
\begin{frame}{Key Envisioned Features of BiCycure}
    \begin{block}{Verification of Ownership}
        \begin{itemize}
            \item RFID tags allow authorities and individuals to quickly verify who is the owner.
            \item Tamper-proof tags reduce theft attempts.
        \end{itemize}
    \end{block}
    
    \begin{block}{Transfer of Ownership}
        \begin{itemize}
            \item Need: smooth and traceable digital ownership transfer.
        \end{itemize}
    \end{block}
\end{frame}

\begin{frame}{Key Envisioned Features of BiCycure (cont'd)}
    \begin{block}{Declaration of Theft}
        \begin{itemize}
            \item Owners can report theft directly in the app.
            \item Public theft declarations and rewards encourage community involvement.
        \end{itemize}
    \end{block}
\end{frame}


% Ethical Considerations
\section{Ethical Considerations}
\begin{frame}{Ethical Considerations and Privacy}
    \begin{itemize}
        \item Large-scale scanning can lead to tracking of bicycles and owners.
        \item System usability must be inclusive, so everyone benefits from the service.
        \item Personal data needs to only become available once a theft takes place.
    \end{itemize}
\end{frame}

% Solution Architecture
\section{Solution Architecture}
\begin{frame}{System Architecture Overview}
    \begin{itemize}
        \item \textbf{RFID Tags}: Unique identifiers for bicycles.
        \item \textbf{Mobile App}: Scans tags, reports thefts, and verifies ownership.
        \item \textbf{Database}: Relational database for storing bicycle and owner information.
        \item \textbf{API Layer}: RESTful API to communicate between the mobile app and database.
    \end{itemize}
\end{frame}


% Roadmap
\section{Roadmap}
\begin{frame}{Proposed Roadmap}
    \begin{itemize}
        \item \textbf{Phase 1}: Proof of concept with a small user base.
        \item \textbf{Phase 2}: Expand RFID usage to major bicycle retailers.
        \item \textbf{Phase 3}: Collaborate with insurance companies and law enforcement.
        \item \textbf{Phase 4}: Scale to a national level, integrating additional features.
    \end{itemize}
\end{frame}


% Conclusion
\section{Conclusion}
\begin{frame}{Conclusion}
    \begin{itemize}
        \item BiCycure offers a comprehensive solution for bicycle theft prevention and recovery.
        \item Aims to modernize bicycle ownership.
        \item Strong partnerships with law enforcement, retailers, and insurers will enhance adoption and success (read our report :).
    \end{itemize}
\end{frame}

% Thank You slide
\begin{frame}
    \centering
    \Huge Thank You! \\
    \normalsize Questions?
\end{frame}

\end{document}

